\documentclass[../tp2.tex]{subfiles}

\begin{document}
O modelo Bell-LaPadula foi inicialmente adotado para o controlo de acessos em assuntos militares e governamentais. Nestes casos, utilizadores e objetos são divididos em diferentes níveis de segurança, isto é, os utilizadores apenas acedem a informação que corresponde ao seu nível de segurança. Duas afirmações de controlo de acesso que correspondem às duas regras em que o modelo de Bell-LaPadula assenta são por exemplo: ``O público não pode ter acesso a dados confidenciais'' e mais um exemplo, ``Dados secretos não podem ser escritos em ficheiros de acesso público''.\par 
É este tipo de divisão para controlar os acessos que o modelo Bell-LaPadula cria, sendo que os níveis de acesso vão de \texttt{público} a \texttt{confidencial} e \texttt{estritamente confidencial}, por exemplo.\par 
Estes diferentes níveis que garantem o controlo de acesso são atribuídos a utilizadores e a objetos. Entenda-se que um utilizador pode ser um ser humano, um computador ou uma organização, e um objeto poderá ser dispositivos de \textit{input/output}, ficheiros e documentos.
\end{document}